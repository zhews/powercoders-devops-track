\documentclass{article}

\usepackage[pdfusetitle]{hyperref}

\title{Curriculum of Powercoders DevOps Track}
\author{zhews}
\date{\today}

\begin{document}

\maketitle
\thispagestyle{empty}

\newpage

\tableofcontents

\newpage

\section{Purpose}

The target group of this track are participants of the Powercoders bootcamp that
have around ten weeks of experience in IT. The goal of the track is to ramp up
the participants with enough resources and knowledge to be able to start an
internship in a DevOps role after the bootcamp.

\newpage

\section{Topics}

The following topics should be covered as part of this track.

\subsection{DevOps}

DevOps and the mindset behind it get explained. Participants should take the
ideology ``You build it, you run it'' with them. The DevOps cycle including the
corresponding phases are mentioned. Steps involved within the different phases
are covered.

\subsection{Plan}

The core principles of agile project management methods are covered. Important
terms required in weekly discussions are explained. The advantages of agile
especially related to software development are known to the participants.

\subsection{Create}

The differences between compiled and interpreted programming languages get
explained through two projects that will be used in the other topics.
Implementing these two projects should lead to understanding the language
specific ways of fetching dependencies, building, testing and running the code.
To properly store, review and contribute code some more advance Git topics are
covered. For the participants to better understand what they are implementing
the most important concepts in system design topics get taught. As the concepts
in system design sometimes require knowledge in networking an introduction to
the relevant topics is given.

\subsection{Verify}

Participants are taught how to continuously test, lint and build their
projects with the help of a build server. The most important terms related
to testing are covered. The importance of a fast feedback loop for all automated
quality assurance is known to the participants.

\subsection{Package}

Participants are taught about the different artifacts that can be produced from
a code base and how to continuously store the artifacts of their projects in an
artifact repository. How the generated artifacts can be used or deployed is
covered.

\newpage

\subsection{Release}

The difference between continuous delivery and deployment is explained.
Participants learn what a deployment is and where an application can be deployed
to. For that they are taught about traditional ways to deploy, virtual machines,
containers and the cloud. They temporarily deploy their applications to all the
mentioned options.

\subsection{Configure}

To be able to operate a deployed application participants are taught about
Linux. The fact that almost everything can be done in the CLI is known to the
participants. Chores within their projects get automated through scripts. The
concept of infrastructure as code gets explained and practically applied for
their projects and different deployment locations.

\subsection{Monitor}

That monitoring is one of the most important aspects in DevOps is known to
the participants. They start collecting metrics, logs and traces of their
projects that get visualized through open source software. They set up alerting
for different errors. The concepts of site reliability engineering are covered.

\newpage

\section{Lessons}

\subsection{Check In}

In this lesson the teachers and participants should get to know each other. This
includes the teacher quickly informing about their career path and what they are
doing at work. Participants should explain why they chose the DevOps track,
what their current understanding of DevOps is and what they did as a project in
Powercoders so far. The ways of communication get explained to the participants.
They know that it is more important for them to grasp the idea behind the things
that are taught than fully understanding a tool because technology constantly
changes. The participants are informed that the use of any generative AIs is
strongly not recommended as it hinders the learning process.

\subsection{Introduction to DevOps}

The introduction should explain the problem of developer teams just throwing
their applications over to the operation teams and letting them take care of the
rest. It should cover that the concept of ``You build it, you ship it'' removes
the wall between dev and ops people. The term shift left and how DevOps helps
to achieve that is taught. The DevOps cycle is shown to the participants and a
small introduction for each phase has been done.

\subsection{Project Introduction}

The two projects that are done as part of this track are shown to the
participants.

\subsection{Agile}

Participants are introduced to the agile manifesto. Why waterfall
based methods do not work in software development gets explained. Common scrum
terms get covered so that they can use them in daily conversations.

\subsection{Tooling around Programming Languages}

This lesson explains common things in most programming languages like fetching
dependencies, testing, building and running with the help of JavaScript and Go.
The differences of these two languages are shown, and their tooling gets
covered.

\newpage

\subsection{Productive Git}

The participants get introduced to trunk based development and the git flow.
They learn the different ways to merge code of different contributors and how to
fix possibly resulting merge conflicts. Common practices like branch protection,
conventional commits, not publishing secrets and git hooks get covered. The
participants know what pull requests are and why they are important. How to
revert or reset changes in case of problems gets explained.

\subsection{Basic System Design}

Participants get to know the difference between monolithic and microservice
based applications. The differences between stateful and stateless applications
get explained. They know about load balancers, proxies, ways of persistence,
caching, rate limiting, queues and how they are applied in practice.

\subsection{Network Fundamentals}

At the beginning common topics like MAC addresses, IP addresses, nodes, switches
routers and ports get covered. In a second step the participants get to know how
these topics correlate to each other. The OSI model gets explained to the
participants with the help of different examples. Different protocols like ICMP,
ARP, TCP, UDP, SMTP, FTP, SSH, DHCP, DNS, HTTP and HTTPS including the related
ports get covered. Common firewall topics like routing, rules and VPNs are known
to the participants.

\subsection{Testing Fundamentals}

The importance of testing is explained to the participants. They learn about the
two main categories of testing (functional and non-functional) and go into more
details regarding unit (white, gray and black box), integration and E2E testing.
The testing pyramid is shown to the participants.

\subsection{CI / CD}

The most typical steps of the CI / CD process get covered. This being checkout,
linting, testing, scanning, building or packaging, delivering and sometimes
deploying. Participants get to know about build serves and how the process can
be automated with the help of different triggers. They learn about the different
artifacts a project can produce and where they can be stored.

\subsection{Introduction to GitHub Actions}

The CI / CD solution GitHub Actions gets introduced to the participants. They
learn how most of the topics discussed in the previous lessons can be
implemented in their solution.

\subsection{Introduction to Jenkins}

The CI / CD solution Jenkins gets introduced to the participants. They learn how
most of the topics discussed in the previous lessons can be implemented in their
solution.

\subsection{Introduction to Nexus}

The artifact hub of Nexus gets introduced to the participants. They know how
artifacts can be uploaded and retrieved from Nexus.

\subsection{Where to Deploy?}

Participants learn about the different places an application can be deployed to.
This includes on premise, virtual machines, containers and the cloud. They learn
about the advantages and disadvantages of each solution and when to prefer what.

\subsection{Server Configuration}

The most important things that have to be configured on every Linux server are
taught to the participants. This includes disk encryption, partitioning,
formatting, network configuration, firewall rules, user accounts, file
permissions and SSH. Besides that they learn more advanced CLI features and get
introduced into bash scripting.

\subsection{Introduction to ProxMox Virtual Environment}

The server management and virtualization platform ProxMox Virtual Environment
gets introduced to the participants. They learn how virtual machines can be
created, configured and then easily redeployed with the help of images.

\subsection{Introduction to Podman}

The container solution Podman gets introduced to the participants. They learn
how existing images containers can be run, new images can be created, networking
gets management, logs can be read, resources can be monitoring, common debugging
tricks and how to interact with container storage.

\subsection{Introduction to Kubernetes}

The container orchestration solution Kubernetes gets introduced to the
participants. They learn about the advantages of Kubernetes, the most basic
resources and how to perform the things learned in the Podman lesson.

\subsection{Introduction to AWS}

The cloud provider AWS gets introduced to the participants. They learn about the
cloud specific solutions of the other mentioned deployment locations and what is
required to get them running.

\subsection{Infrastructure as Code}

This lesson explains the concepts of infrastructure as code, why it is important
to have replicable architecture and what the repository being the single source
of truth means.

\subsection{Introduction to Ansible}

The IT automation system Ansible gets introduced to the participants. They learn
how to automate the things previously done manually in the server configuration
lesson with the help of Ansible.

\subsection{Introduction to OpenTofu}

The infrastructure as code tool OpenTofu gets introduced to the participants.
They learn how to automate the things previously done manually in the AWS lesson
with the help of OpenTofu.

\subsection{Observability}

In this lesson the participants learn about the different components of
observability. Commonly used metrics and how they are visualized, structured
logging and why it is important and how tracing gives clear insights on
bottlenecks.

\subsection{Introduction to Prometheus}

The monitoring solutions Prometheus gets introduced to the participants. They
learn how they can create custom metrics, pull them in Prometheus and query
them.

\subsection{Introduction to Grafana}

The monitoring visualization solution Grafana gets introduced to the
participants. They learn how they can create dashboards visualizing the most
important metrics of their applications.

\subsection{Introduction to Alertmanager}

The alerting solution Alertmanager gets introduced to the participants. They
learn how they can inform themselves via e-mail or slack in case something goes
wrong.

\subsection{Introduction to Tempo}

The tracing solution Tempo gets introduced to the participants. They learn how
they can trace important parts in their application and inspect them in Tempo.

\subsection{Introudction to Loki}

The log aggregation system Loki gets introduced to the participants. They learn
how logs should be structured to easily query them with Loki.

\subsection{Site Reliability Engineering}

The concept of site reliability engineering gets introduced to the participants.
The difference between DevOps and SRE gets covered and terms commonly used like
SLIs, SLOs and SLAs get explained.

\newpage

\section{Exercises}

\newpage

\section{Resources}

\newpage

\section{Timeline}

The track takes place over the time of three and a half weeks. Within this time
the participants should spend around forty percent of their time in DevOps
related classes and sixty percent doing hands on exercises that correlate to the
theory learned in the classes.

\end{document}
