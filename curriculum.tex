\documentclass{article}

\usepackage[pdfusetitle]{hyperref}

\title{Curriculum of Powercoders DevOps Track}
\author{zhews}
\date{\today}

\begin{document}

\maketitle
\thispagestyle{empty}

\newpage

\tableofcontents

\newpage

\section{Purpose}

The target group of this track are participants of the Powercoders bootcamp that
have around ten weeks of experience in IT. The goal of the track is to ramp up
the participants with enough resources and knowledge to be able to start an
internship in a DevOps role after the bootcamp.

\section{Topics}

The following topics should be covered as part of this track.

\subsection{DevOps}

DevOps and the mindset behind it get explained. Participants should take the
ideology ``You build it, you run it'' with them. The DevOps cycle including the
corresponding phases are mentioned. Steps involved within the different phases
are covered.

\subsection{Plan}

The core principles of agile project management methods are covered. Important
terms required in weekly discussions are explained. The advantages of agile
especially related to software development are known to the participants.

\subsection{Create}

The differences between compiled and interpreted programming languages get
explained through two projects that will be used in the other topics.
Implementing these two projects should lead to understanding the language
specific ways of fetching dependencies, building, running and testing the code.
To properly store, review and contribute code some more advance Git topics are
covered. For the participants to better understand what they are implementing
the most important concepts in system design topics get taught. As the concepts
in system design sometimes require knowledge in networking an introduction to
the relevant topics is given.

\subsection{Verify}

Participants are taught how to continuously test, lint and build their
projects with the help of a build server. The most important terms related
to testing are covered. The importance of a fast feedback loop for all automated
quality assurance is known to the participants.

\subsection{Package}

Participants are taught about the different artifacts that can be produced from
a code base and how to continuously store the artifacts of their projects in an
artifact repository. How the generated artifacts can be used or deployed is
covered.

\subsection{Release}

The difference between continuous delivery and deployment is explained.
Participants learn what a deployment is and where an application can be deployed
to. For that they are taught about traditional ways to deploy, virtual machines,
containers and the cloud. They temporarily deploy their applications to all the
mentioned options.

\subsection{Configure}

To be able to operate a deployed application participants are taught about
Linux. The fact that almost everything can be done in the CLI is known to the
participants. Chores within their projects get automated through scripts. The
concept of infrastructure as code gets explained and practically applied for
their projects and different deployment locations.

\subsection{Monitor}

That monitoring is one of the most important aspects in DevOps is known to
the participants. They start collecting metrics, logs and traces of their
projects that get visualized through open source software. They set up alerting
for different errors. The concepts of site reliability engineering are covered.

\section{Lessons}

\section{Exercises}

\section{Assessments}

\section{Timeline}

The track takes place over the time of three and a half weeks. Within this time
the participants should spend around forty percent of their time in DevOps
related classes and sixty percent doing hands on exercises that correlate to the
theory learned in the classes.

\end{document}
